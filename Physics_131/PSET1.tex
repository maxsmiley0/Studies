\documentclass[12pt]{article}
\usepackage{geometry, amsfonts, parskip, amssymb, amsmath}
\pagestyle{empty}

\begin{document}
\title{Physics 131 Problem Set 1}
\author{Max Smiley}
\date{Spring 2021}
\maketitle

1). To find the rank of the following matrices, we will put them into reduced row echelon form. 
$$\begin{pmatrix}
1 & 1 & 2 \\
2 & 4 & 6 \\
3 & 2 & 5
\end{pmatrix} 
\begin{matrix}
\\
-2 \times I \\ 
-3 \times I \\
\end{matrix}
\Rightarrow
\begin{pmatrix}
1 & 1 & 2 \\
0 & 2 & 2 \\
0 & -1 & -1
\end{pmatrix}
\begin{matrix}
\\
+ III\\
+ \frac{1}{2} \times II
\end{matrix}
\Rightarrow
\begin{pmatrix}
1 & 1 & 2\\
0 & 1 & 1\\
0 & 0 & 0\\
\end{pmatrix}
\begin{matrix}
- II\\
\\
\\
\end{matrix}
\Rightarrow
\begin{pmatrix}
1 & 0 & 1\\
0 & 1 & 1\\
0 & 0 & 0
\end{pmatrix}
$$\\
This matrix has 2 pivots, and thus has a rank of 2.
$$
\begin{pmatrix}
2 & -3 & 5 & 3\\
4 & -1 & 1 & 1\\
3 & -2 & 3 & 4
\end{pmatrix}
\begin{matrix}
\\
-2 \times I\\
-\frac{3}{2} \times I
\end{matrix}
\Rightarrow
\begin{pmatrix}
2 & -3 & 5 & 3\\
0 & 5 & -9 & -5\\
0 & \frac{5}{2} & -\frac{9}{2} & -\frac{1}{2}
\end{pmatrix}
\begin{matrix}
\\
\\
\times 2 - II
\end{matrix}
\Rightarrow
\begin{pmatrix}
2 & -3 & 5 & 3\\
0 & 5 & -9 & -5\\
0 & 0 & 0 & 4
\end{pmatrix}
\begin{matrix}
\times \frac{1}{2}\\
\times \frac{1}{5}\\
\times \frac{1}{4}
\end{matrix}
\Rightarrow
$$
$$
\begin{pmatrix}
1 & -\frac{3}{2} & \frac{5}{2} & \frac{3}{2}\\
0 & 1 & -\frac{9}{5} & -1\\
0 & 0 & 0 & 1
\end{pmatrix}
\begin{matrix}
+ \frac{3}{2} II\\
+ III\\
\\
\end{matrix}
\Rightarrow
\begin{pmatrix}
1 & 0 & -\frac{1}{5} & 0\\
0 & 1 & -\frac{9}{5} & 0\\
0 & 0 & 0 & 1
\end{pmatrix}
$$
This matrix has 3 pivots, and thus has a rank of 3.\\

2). Let the general $2 \times 2$ matrix  have the representation 
$\begin{pmatrix}
x & y\\
z & w
\end{pmatrix}$.
If its square is the zero matrix, then 
$$\begin{pmatrix}
x & y\\
z & w
\end{pmatrix}
\begin{pmatrix}
x & y\\
z & w
\end{pmatrix}
 = 
 \begin{pmatrix}
x^2 + y z & x y + y w\\
x z + z w & y z + w^2
\end{pmatrix}
=
\begin{pmatrix}
0 & 0\\
0 & 0
\end{pmatrix}
\Rightarrow
\begin{matrix}
x^2 + yz = 0\\
xy + yw = 0\\
xz + zw = 0\\
yz + w^2 = 0
\end{matrix}
$$
Assuming not both $y, z = 0$, we have $\begin{matrix}
x^2 = -y z\\
w^2 = -y z\\
x = -w\\
\end{matrix}
\Rightarrow
\begin{matrix}
x = \sqrt{-y z}\\
w = -\sqrt{-y z}
\end{matrix}
\Rightarrow
\begin{pmatrix}
x & y\\
x & w
\end{pmatrix}
=
\begin{pmatrix}
\sqrt{-yz} & y\\
z & -\sqrt{-yz}
\end{pmatrix}
$. Now if we choose 
$\begin{matrix}
a = \sqrt{-z}\\
b = \sqrt{y}
\end{matrix}$, we see our arbitrary matrix can be represented as
$$
\begin{pmatrix}
ab & b^2\\
-a^2 & -ab 
\end{pmatrix}
$$\\
Note that if both $y, z = 0$, this would still hold, as the arbitrary matrix representation would be $\begin{pmatrix}
x & 0\\
0 & w
\end{pmatrix}
$, giving us
$$
\begin{pmatrix}
x & 0\\
0 & w
\end{pmatrix}
\begin{pmatrix}
x & 0\\
0 & w
\end{pmatrix}
=
\begin{pmatrix}
x^2 & 0\\
0 & w^2
\end{pmatrix}
=
\begin{pmatrix}
0 & 0\\
0 & 0
\end{pmatrix}
\Rightarrow
\begin{matrix}
x = 0\\
w = 0\\
\end{matrix}
$$ and we could trivially choose $a = b = 0$.\\

3.) Suppose we have $A = \begin{pmatrix}
1 & 0\\
0 & 1
\end{pmatrix}$ and $B = \begin{pmatrix}
1 & 0\\
0 & 1
\end{pmatrix}$. Then $det(A) = det(B) = 1\Rightarrow det(A) + det(B) = 2$. However, $det(A + B) = det\begin{pmatrix}
2 & 0\\
0 & 2
\end{pmatrix} = 4 \neq 2$, hence, this property does not hold in general.\\

4.) If two nonzero vectors lie in a plane, then their cross product will produce a vector normal to the plane. If $\vec{a}, \vec{b}, \vec{c}, \vec{d}$ are all in the same plane, then $\vec{e} = \vec{a} \times \vec{b}$ and $\vec{f} = \vec{c} \times \vec{d}$ will both be normal vectors to the plane, i.e. pointing in the same direction. Since $\vec{e} \times \vec{f} := |\vec{e}||\vec{f}| \sin{\theta}$, where $\theta = 0$ as both vectors are pointing in the same direction, $\vec{e} \times \vec{f} = 0 \Rightarrow (\vec{a} \times \vec{b}) \times (\vec{c} \times \vec{d}) = 0$. Note if any $\vec{a}, \vec{b}, \vec{c}, \vec{d} = 0$, then their cross product with any other vector would be the zero vector, and so the whole product would also still be the zero vector.\\

5.) \\
a.)
$$\sigma_1 \sigma_1 = \begin{pmatrix} 0 & 1 \\ 1 & 0 \end{pmatrix} \begin{pmatrix} 0 & 1 \\ 1 & 0 \end{pmatrix} = \begin{pmatrix} 1 & 0 \\ 0 & 1 \end{pmatrix} = \textbf{1} \qquad \qquad \delta_{11}\textbf{1} + i \epsilon_{11k} \sigma_k = \textbf{1}$$

$$\sigma_1 \sigma_2 = \begin{pmatrix} 0 & 1 \\ 1 & 0 \end{pmatrix} \begin{pmatrix} 0 & -i \\ i & 0 \end{pmatrix} = \begin{pmatrix} i & 0 \\ 0 & -i \end{pmatrix} = i \sigma_3 \qquad \qquad \delta_{12}\textbf{1} + i \epsilon_{123} \sigma_3 = i \sigma_3$$

$$\sigma_1 \sigma_3 = \begin{pmatrix} 0 & 1 \\ 1 & 0 \end{pmatrix} \begin{pmatrix} 1 & 0 \\ 0 & -1 \end{pmatrix} = \begin{pmatrix} 0 & -1 \\ 1 & 0 \end{pmatrix} = -i \sigma_2 \qquad \qquad \delta_{13}\textbf{1} + i \epsilon_{132} \sigma_2 = -i \sigma_2$$

$$\sigma_2 \sigma_1 = \begin{pmatrix} 0 & -i \\ i & 0 \end{pmatrix} \begin{pmatrix} 0 & 1 \\ 1 & 0 \end{pmatrix} = \begin{pmatrix} -i & 0 \\ 0 & i \end{pmatrix} = -i \sigma_3 \qquad \qquad \delta_{21}\textbf{1} + i \epsilon_{213} \sigma_3 = -i \sigma_3$$

$$\sigma_2 \sigma_2 = \begin{pmatrix} 0 & -i \\ i & 0 \end{pmatrix} \begin{pmatrix} 0 & -i \\ i & 0 \end{pmatrix} = \begin{pmatrix} 1 & 0 \\ 0 & 1 \end{pmatrix} = \textbf{1} \qquad \qquad \delta_{22}\textbf{1} + i \epsilon_{22k} \sigma_k = \textbf{1}$$

$$\sigma_2 \sigma_3 = \begin{pmatrix} 0 & -i \\ i & 0 \end{pmatrix} \begin{pmatrix} 1 & 0 \\ 0 & -1 \end{pmatrix} = \begin{pmatrix} 0 & i \\ i & 0 \end{pmatrix} = i \sigma_1 \qquad \qquad \delta_{23}\textbf{1} + i \epsilon_{231} \sigma_1 = i \sigma_1$$

$$\sigma_3 \sigma_1 = \begin{pmatrix} 1 & 0 \\ 0 & -1 \end{pmatrix} \begin{pmatrix} 0 & 1 \\ 1 & 0 \end{pmatrix} = \begin{pmatrix} 0 & 1 \\ -1 & 0 \end{pmatrix} = i \sigma_2 \qquad \qquad \delta_{31}\textbf{1} + i \epsilon_{312} \sigma_2 = i \sigma_2$$

$$\sigma_3 \sigma_2 = \begin{pmatrix} 1 & 0 \\ 0 & -1 \end{pmatrix} \begin{pmatrix} 0 & -i \\ i & 0 \end{pmatrix} = \begin{pmatrix} 0 & -i \\ -i & 0 \end{pmatrix} = -i \sigma_1 \qquad \qquad \delta_{32}\textbf{1} + i \epsilon_{321} \sigma_1 = -i \sigma_1$$

$$\sigma_3 \sigma_3 = \begin{pmatrix} 1 & 0 \\ 0 & -1 \end{pmatrix} \begin{pmatrix} 1 & 0 \\ 0 & -1 \end{pmatrix} = \begin{pmatrix} 1 & 0 \\ 0 & 1 \end{pmatrix} = \textbf{1} \qquad \qquad \delta_{33}\textbf{1} + i \epsilon_{33k} \sigma_k = \textbf{1}$$

b.) Let $\vec{A} = (a_1, a_2, a_3)$, $\vec{B} = (b_1, b_2, b_3)$, and $\vec{\sigma} = (\sigma_1, \sigma_2, \sigma_3)$. Then
$$(\sigma_1 \cdot \vec{A})(\sigma_1 \cdot \vec{B}) =  (a_1 \sigma_1 + a_2 \sigma_2 + a_3 \sigma_3)(b_1 \sigma_1 + b_2 \sigma_2 + b_3 \sigma_3) = \sum_{ij}^3 a_{i} b_{j} \sigma_i \sigma_j$$
$$\Rightarrow \sum_{ij}^3 a_i b_j (\delta_{ij} \textbf{1} +  i \epsilon_{ijk} \sigma_k) = (a_1 b_1 + a_2 b_2 + a_3 b_3) \textbf{1} + i \sum_{ij}^3 a_i b_j \epsilon_{ijk} \sigma_k$$
$$\Rightarrow (\vec{A} \cdot \vec{B}) \textbf{1} + i(a_1 b_2 \sigma_3 - a_1 b_3 \sigma_2 - a_2 b_1 \sigma_3 + a_2 b_3 \sigma_1 + a_3 b_1 \sigma_2 - a_3 b_2 \sigma_1)$$
$$\Rightarrow (\vec{A} \cdot \vec{B}) \textbf{1} + i(\sigma_1(a_2 b_3 - a_3 b_2) + \sigma_2(-a_1 b_3 + a_3 b_1) + \sigma_3(a_1 b_2 - a_2 b_1))$$
$$\Rightarrow (\vec{A} \cdot \vec{B}) \textbf{1} + i(a_2 b_3 - a_3 b_2, -a_1 b_3 + a_3 b_1, a_1 b_2 - a_2 b_1) \cdot \vec{\sigma})$$
$$\Rightarrow (\vec{A} \cdot \vec{B}) \textbf{1} + i(\vec{A} \times \vec{B}) \cdot \vec{\sigma}$$

c.)$$\sin(k \sigma_1) = k \sigma_1 + \frac{k^3 \sigma_1^3}{3!} + \frac{k^5 \sigma_1^5}{5!} + \dots$$
Since $\sigma_1^2 = \textbf{1}$, then any odd power of $\sigma_1$ is $\sigma_1$, i.e. $\sigma_1^{2n + 1} = \sigma_1^{2n} \sigma_1 = \textbf{1} \sigma_1 = \sigma_1$.
$$\Rightarrow k \sigma_1 + \frac{k^3 \sigma_1}{3!} + \frac{k^5 \sigma_1}{5!} + \dots$$
$$\Rightarrow \sigma_1 (k + \frac{k^3}{3!} + \frac{k^5}{5!} + \dots)$$
$$\Rightarrow \sigma_1 \sin(k) = \begin{pmatrix} 0 & \sin(k) \\ \sin(k) & 0 \end{pmatrix}$$

$$e^{k \sigma_3} = \textbf{1} + k \sigma_3 + \frac{k^2 \sigma_3^2}{2!} + \frac{k^3 \sigma_3^3}{3!} + \dots$$
Since $\sigma_3^2 = \textbf{1}$, all even powers of $\sigma_3$ equal one, i.e. $\sigma_3^{2n} = \textbf{1}$, and all odd powers of $\sigma_1$  equal $\sigma_1$, i.e. $\sigma_3^{2n + 1} = \sigma_3^{2n} \sigma_3 = \textbf{1} \sigma_3 = \sigma_3$ 
$$\Rightarrow \textbf{1} + k \sigma_3 + \frac{k^2 \textbf{1}}{2!} + \frac{k^3 \sigma_3}{3!} + \dots$$
$$\Rightarrow (\textbf{1} + \frac{k^2 \textbf{1}}{2!} + \dots) + (k \sigma_3 + \frac{k^3 \sigma_3}{3!} + \dots)$$
$$\Rightarrow \textbf{1}(1 + \frac{k^2}{2!} + \dots) + \sigma_3(k + \frac{k^3}{3!} + \dots)$$
$$\Rightarrow \cosh(k) \textbf{1} + \sinh(k) \sigma_3 = \begin{pmatrix} \cosh(k) + \sinh(k) & 0 \\ 0 & \cosh(k) - \sinh(k)\end{pmatrix}$$

$$e^{\theta \sigma'} = \textbf{1} + \theta \sigma' + \frac{\theta^2 \sigma'^2}{2!} + \frac{\theta^3 \sigma'^3}{3!} + \dots$$
Where $\sigma' = i \sigma_2 = \begin{pmatrix} 0 & 1\\ -1 & 0 \end{pmatrix}$. Note $\sigma'^2 = -\textbf{1}$, and hence $\sigma'^3 = -\sigma'$, $\sigma'^4 = \textbf{1}$, and the sigmas subsequently cycle through these values every four powers.
$$\Rightarrow \textbf{1} + \theta \sigma' - \frac{\theta^2 \textbf{1}}{2!} - \frac{\theta^3 \sigma'}{3!} + \dots$$
$$\Rightarrow (\textbf{1} - \frac{\theta^2 \textbf{1}}{2!} + \dots) + (\theta \sigma' - \frac{\theta^3 \sigma'}{3!} + \dots)$$
$$\Rightarrow \textbf{1}(1 - \frac{\theta^2}{2!} + \dots) + \sigma'(\theta - \frac{\theta^3}{3!} + \dots)$$
$$\Rightarrow \cos(\theta) \textbf{1} + \sin(\theta) \sigma'$$
$$\Rightarrow \begin{pmatrix} \cos(\theta) & 0 \\ 0 & \cos(\theta) \end{pmatrix} + \begin{pmatrix} 0 & \sin(\theta) \\ -\sin(\theta) & 0 \end{pmatrix}$$
$$\Rightarrow \begin{pmatrix} \cos(\theta) & \sin(\theta) \\ -\sin(\theta) & \cos(\theta) \end{pmatrix}$$

This is a rotation matrix that will rotate a vector in $\mathbb{R}^2$ by $-\theta$.

6a.) $det(A) = det\begin{pmatrix} 0 & -1 \\ -2 & 0 \end{pmatrix} + det\begin{pmatrix} 4 & -1 \\ 4 & 0 \end{pmatrix} + det\begin{pmatrix} 4 & 0 \\ 4 & -2\end{pmatrix}$ = $-6$. 
$$A = \begin{pmatrix} 1 & -1 & 1 \\ 4 & 0 & -1 \\ 4 & -2 & 0 \end{pmatrix} \Rightarrow M = \begin{pmatrix} -2 & 4 & -8 \\ 2 & -4 & 2 \\ 1 & -5 & 4\end{pmatrix} \Rightarrow C = \begin{pmatrix} -2 & -4 & -8 \\ -2 & -4 & -2 \\ 1 & 5 & 4\end{pmatrix}$$
$$\Rightarrow A^{-1} = \frac{1}{-6} \begin{pmatrix} -2 & -2 & 1 \\ -4 & -4 & 5 \\ -8 & -2 & 4\end{pmatrix} = \begin{pmatrix} \frac{1}{3} & \frac{1}{3} & \frac{-1}{6} \\ \frac{2}{3} & \frac{2}{3} & \frac{-5}{6} \\ \frac{4}{3} & \frac{1}{3} & \frac{-2}{3} \end{pmatrix}$$

$det(B) = det\begin{pmatrix} 1 & 1 \\ 1 & 2 \end{pmatrix} +  det\begin{pmatrix} 2 & 1 \\ 2 & 1\end{pmatrix}$ = $1$. 
$$B = \begin{pmatrix} 1 & 0 & 1 \\ 2 & 1 & 1 \\ 2 & 1 & 2 \end{pmatrix} \Rightarrow M = \begin{pmatrix} 1 & 2 & 0 \\ -1 & 0 & 1 \\ -1 & -1 & 1\end{pmatrix} \Rightarrow C = \begin{pmatrix} 1 & -2 & 0 \\ 1 & 0 & -1 \\ -1 & 1 & 1\end{pmatrix}$$
$$\Rightarrow B^{-1} = \begin{pmatrix} 1 & 1 & -1 \\ -2 & 0 & 1 \\ 0 & -1 & 1\end{pmatrix}$$
$$B^{-1}AB = \begin{pmatrix} 1 & 1 & -1 \\ -2 & 0 & 1 \\ 0 & -1 & 1\end{pmatrix} \begin{pmatrix} 1 & -1 & 1 \\ 4 & 0 & -1 \\ 4 & -2 & 0 \end{pmatrix} \begin{pmatrix} 1 & 0 & 1 \\ 2 & 1 & 1 \\ 2 & 1 & 2 \end{pmatrix}= \begin{pmatrix} 1 & 1 & -1 \\ -2 & 0 & 1 \\ 0 & -1 & 1\end{pmatrix}\begin{pmatrix} 1 & 0 & 2 \\ 2 & -1 & 2 \\ 0 & -2 & 2\end{pmatrix}$$ 
$$ = \begin{pmatrix} 3 & 1 & 2 \\ -2 & -2 & -2 \\ -2 & -1 & 0\end{pmatrix}$$

$$B^{-1}A^{-1}B = \begin{pmatrix} 1 & 1 & -1 \\ -2 & 0 & 1 \\ 0 & -1 & 1\end{pmatrix} \begin{pmatrix} \frac{1}{3} & \frac{1}{3} & \frac{-1}{6} \\ \frac{2}{3} & \frac{2}{3} & \frac{-5}{6} \\ \frac{4}{3} & \frac{1}{3} & \frac{-2}{3} \end{pmatrix} \begin{pmatrix} 1 & 0 & 1 \\ 2 & 1 & 1 \\ 2 & 1 & 2 \end{pmatrix}= \begin{pmatrix} -\frac{1}{3} & \frac{2}{3} & -\frac{1}{3} \\ \frac{2}{3} & -\frac{1}{3} & -\frac{1}{3} \\ \frac{2}{3} & -\frac{1}{3} & \frac{1}{6}\end{pmatrix}\begin{pmatrix} 1 & 0 & 2 \\ 2 & -1 & 2 \\ 0 & -2 & 2\end{pmatrix}$$ 
$$ = \begin{pmatrix} \frac{1}{3} & \frac{1}{3} & -\frac{1}{3} \\ -\frac{2}{3} & -\frac{2}{3} & -\frac{1}{3} \\ \frac{1}{3} & -\frac{1}{6} & \frac{2}{3}\end{pmatrix}$$

b.) $(B^{-1}AB)(B^{-1}A^{-1}B) = \begin{pmatrix} 3 & 1 & 2 \\ -2 & -2 & -2 \\ -2 & -1 & 0\end{pmatrix} \begin{pmatrix} \frac{1}{3} & \frac{1}{3} & -\frac{1}{3} \\ -\frac{2}{3} & -\frac{2}{3} & -\frac{1}{3} \\ \frac{1}{3} & -\frac{1}{6} & \frac{2}{3}\end{pmatrix} = \begin{pmatrix} 1 & 0 & 0 \\ 0 & 1 & 0 \\ 0 & 0 & 1 \end{pmatrix} = \textbf{1}$ and $(B^{-1}A^{-1}B)(B^{-1}AB) = \begin{pmatrix} \frac{1}{3} & \frac{1}{3} & -\frac{1}{3} \\ -\frac{2}{3} & -\frac{2}{3} & -\frac{1}{3} \\ \frac{1}{3} & -\frac{1}{6} & \frac{2}{3}\end{pmatrix}  \begin{pmatrix} 3 & 1 & 2 \\ -2 & -2 & -2 \\ -2 & -1 & 0\end{pmatrix} = \begin{pmatrix} 1 & 0 & 0 \\ 0 & 1 & 0 \\ 0 & 0 & 1 \end{pmatrix} = \textbf{1}$ and thus $(B^{-1}AB)$ and $(B^{-1}A^{-1}B)$ are inverses of each other.\\

Now we claim the inverse of a product of $n$ matricies $A_1, \dots A_n$ is equal to the product inverse of the matrices in reverse order, namely $(A_1 \dots A_n)^{-1} = A_n^{-1} \dots A_1^{-1}$. (where we assume all the inverse of $A_i$ exists $\forall i$)\\

\textbf{Proof By Induction: }Let $P(n)$ be the preceding statement.\\
Base Case: $n = 2$.\\
Suppose we have the equation $A_1 A_2 x = y$. Then $B = (A_1 A_2)^{-1}$ is the unique matrix such that $x = By$. If we multiply both sides (on the left) first by $A_1^{-1}$, and then $A_2^{-1}$, we get
$$A_2^{-1} A_1^{-1} A_1 A_2 x = A_2^{-1} A_1^{-1} y \Rightarrow A_2^{-1} A_2 x = A_2^{-1} A_1^{-1} y \Rightarrow x= A_2^{-1} A_1^{-1} y$$
Thus, $B = (A_1 A_2)^{-1} = A_2^{-1} A_1^{-1}$\\

Induction Step: Assume $P(n - 1)$\\
Suppose now we have some equation $(\prod_{i = 1}^n A_i) x = y$ that involves the product of $n$ matrices. Then $B = (\prod_{i = 1}^n A_i)^{-1}$ is the unique matrix such that $x = By$. By matrix associativity, we have $$(\prod_{i = 1}^n A_i) x = y = A_1 (\prod_{i = 2}^n A_i) x = y \Rightarrow (\prod_{i = 2}^n A_i) x = A_1^{-1} y$$
However, since $(\prod_{i = 2}^n A_i)$ is simply a product of $n - 1$ matrices, by the induction hypothesis, its inverse is $A_n^{-1} \dots A_2^{-1}$.
$$\Rightarrow x = A_n^{-1} \dots A_1^{-1} y$$
$$\Rightarrow B = (A_1 \dots A_n)^{-1} = A_n^{-1} \dots A_1^{-1}$$














\end{document}